\documentclass[final,hyperref={pdfpagelabels=false}]{beamer}
\mode<presentation>
{
	%  \usetheme{Berlin}
	%\usetheme{Dreuw}
}
%\usepackage{fontspec}
\usepackage{times}
\usepackage{amsmath,amsthm, amssymb, latexsym}
%\boldmath
\usepackage[english]{babel}
\usepackage[utf8]{inputenc}
\usepackage[orientation=portrait,size=a4,scale=0.3]{beamerposter}

%%%%%%%%%%%%%%%%%%%%%%%%%%%%%%%%%%%%%%%%%%%%%%%%%%%%%%%%%%%%%%%%%%%%%%%%%%%%%%%%%5
\graphicspath{{figures/}}
%\title{}
%\author[]{Teo}
%\institute{Inf}
%\date{}
%\newenvironment{oppgave}
%{\footnotesize\raggedright}
%{\normalsize}
\newcommand{\oppgave}[2]
{\center\normalsize Oppgave\\
	{\footnotesize\raggedright \textit{#1}\\#2\\}}
\newcommand{\deloppgave}[2]
{\footnotesize\center\textbf{Deloppgave}\\ \raggedright \textit{#1}\\#2\\}
\newcommand{\matte}[1]{\vspace{-3mm}\begin{align*}#1\end{align*}}
\newcommand{\maths}[1]{\vspace{-3mm}\begin{align*}#1\end{align*}\\\vspace{-3mm}}
\newcommand{\ol}[0]{\overline}
\newcommand{\tema}[2]{\footnotesize\raggedright\textbf{#1}\\{#2}\par}
\newcommand{\body}[1]{\footnotesize\raggedright{#1}\par}
\newcommand{\subsect}[1]{\center\normalsize{#1}\\}

%%%%%%%%%%%%%%%%%%%%%%%%%%%%%%%%%%%%%%%%%%%%%%%%%%%%%%%%%%%%%%%%%%%%%%%%%%%%%%%%%5
\begin{document}
\begin{frame}{} 
	\begin{columns}[t]
		\begin{column}{.25\linewidth}
			\begin{block}{\center\normalsize Estimering}
				\center\normalsize Viktige estimatoregenskaper:\\
				{\footnotesize\raggedright
					Estimatoren $\hat\Theta$ bør være \textbf{forventningsrett}, DVS $E(\hat\Theta)=\Theta$, hvor $\Theta$ er et parameter du prøver å estiemre. \\
					\textbf{Variansen} til $\hat\Theta$ bør være synkende med økende antall observasjoner.\\
					Om du har to estimatorer $\hat\Theta_1, \hat\Theta_2$ er estimatoren med minst varians den mest effektive estimatoren for $\Theta$.
				}
				\center\normalsize Noen vanlige estimatorer, standardsituasjoner:\\
				{\footnotesize\raggedright
					$\mu$: For et tilfeldig utvalg av størrelse $n$ fra en populasjon med forventning $\mu$ og varians $\sigma^2$ er en estimator for $\mu$ gitt ved: \\
					$\bar X=\frac{1}{n}\sum_{i=1}^nX_i\quad E(\bar X)=\mu\quad Var(\bar X)=\frac{\sigma^2}{n}$\\
					$\sigma^2$: For et tilfeldig utvalg av størrelse $n$ fra en populasjon med forventning $\mu$ og varians $\sigma^2$ er en estimator for $\sigma^2$ gitt ved: \\
					$S^2=\frac{1}{n-1}\sum_{i=1}^n(X_i-\bar X)^2\,E(S^2)=\sigma^2\,Var(S^2)=\frac{2\sigma^4}{n-1}$\\
					$p$: For et tilfeldig utvalg av størrelse $n$ fra et binomisk forsøk (Bernulli-forsøksrekke) med sannsynlighet $p$. En estimator for $p$ er gitt ved\\
					$\hat p=\frac{X}{n}\qquad E(\hat p)=p\qquad Var(\hat p)=\frac{p(1-p)}{n}$\\
					$\mu_1-\mu_2$: For to uavhengige utvalg av størrelser $n_1,n_2$ fra populasjoner med forventning $\mu_1,\mu_2$ og varianser $\sigma_1^2,\sigma_2^2$ er en estimator for $\mu_1-\mu_2$ gitt ved \\
					$\bar X_1 - \bar X_2 \, E(\bar X_1 - \bar X_2)=\mu_1-\mu_2\,Var(\bar X_1 - \bar X_2)=\frac{\sigma_1^2}{n_1}+\frac{\sigma_2^2}{n_2}$\\
					$\frac{\sigma_1^2}{\sigma_2^2}$: For to tilfeldig utvalg av størrelser $n_1,n_2$ fra normalfordelte populasjoner med forventninger $\mu_1,\mu_2$ of varianser $\sigma_1^2,\sigma_2^2$ er en estimator for $\frac{\sigma_1^2}{\sigma_2^2}$ gitt ved:
					$\frac{S_1^2}{S_2^2}$\\
					$p_1-p_2$: For to uavhengige utvalg fra binomiske forsøk med sannsynligheter $p_1,p_2$ er en estimator for $p_1-p_2$ gitt ved:\\
					$\hat p_1 - \hat p_2 = \frac{X_1}{n_1}-\frac{X_2}{n_2}\,E(\hat p_1 - \hat p_2)=p_1-p_2\,$\\
					$Var(\hat p_1 - \hat p_2)=\frac{\hat p_1 (1 - \hat p_1)}{n_1} + \frac{\hat p_2 (1 - \hat p_2)}{n_2}$\\
					$\mu_D$: For to parvise tilfeldig utvalg av størrlese $n$ der differansene fra populasjonene med forventning $\mu_D$ og varians $\sigma_D^2$ er en estimator for $\mu_D$ gitt ved: \\
					$\bar D \qquad E(\bar D)=\mu_D \qquad Var(\bar D)=\frac{\sigma_D^2}{n}$
				}
			\end{block}
			\begin{block}{\center\normalsize Utvalgsfordelinger}
				{\footnotesize\raggedright
					En utvalgsfordeling er fordelinga for en observator (funksjon av de stokastiske variablene i utvalget) for et (tilfeldig) utvalg data. Vi er gjerne interresert i fordelinga til (de spesielle observatorene som er ) estimatorer for paramerere i populasjonen, ofte på standarisert form.
				}
				\center\normalsize Standardsituasjonene (Utvalgsfordelinger)\\
				{\footnotesize\raggedright
					$\bar X, Z$: For et tilfeldig utvalg av størrelse $n$ fra en normalfordelt populasjon med forventning $\mu$ og varians $\sigma^2$ vil:\\
					$\bar X \sim N \left( \mu, \frac{\sigma}{ \sqrt n} \right) \qquad Z=\frac{\bar{X}-E(\bar X)}{\sqrt{Var(\bar X)}}=\frac{\bar X - \mu}{\frac{\sigma}{\sqrt{n}}}\sim N(0,1)$\\
					\textbf{SGT}: Selv om populasjonen ikke er normalfordelt vil resultatet over gjelde når $n \rightarrow \infty$. Regner vanlighvis tilnærminga for god når $n\geq 30$\\
					$T$: For et tilfeldig utvalg av størrelser $n$ fra en normalfordelt populasjon med forventning $\mu$ og varians $\sigma^2$, der variansen estimeres ved $S^2$ fra utvalget har vi at:\\
					$T=\frac{\bar X - \mu}{\frac{S}{\sqrt n}}=\frac{\frac{\bar X - \mu}{\frac{\sigma}{\sqrt n}}}{\sqrt \frac{S^2}{\sigma^2}}\sim t_{n-1}$\\
					$S^2$: For et tilfeldig utvalg av sørrelse $n$ fra en normalfordelt populasjon med forventning $\mu$ og varians $\sigma^2$:\\
					$\frac{(n-1)S^2}{\sigma^2}=\frac{1}{\sigma^2}\sum_{i=1}^n(X_i-\bar X)^2\sim \chi^2_{n-1}$\\
					$\hat p$: For et tilfeldig utvalg av størrelse $n$ fra et binomisk forsøk med sannsynlighet $p$ har vi tilnærmet at:\\
					$Z=\frac{\hat p - E(\hat p)}{\sqrt{Var(\hat p)}}=\frac{\hat p - p}{\sqrt{\frac{p(1-p)}{n}}}\sim N(0,1)$ \\
					$\bar X_1 - \bar X_2$: For to uavhengige tilfeldig utvalg $n_1,n_2$ (normalfordelt) med forventning $\mu_1,\mu_2$ og varianser $\sigma_1^2,\sigma_2^2$ vil: \\
					$\bar X_1 - \bar X_2 \sim N\left( \mu_1-\mu_2,\sqrt{\frac{\sigma_1^2}{n_1}+\frac{\sigma_2^2}{n_2}} \right)$\\
					Eller\\
					$Z=\frac{(\bar X_1 - \bar X_2)-(\mu_1-\mu_2)}{\sqrt{\frac{\sigma_1^2}{n_1}+\frac{\sigma_2^2}{n_2}}}\sim N(0,1)$\\
					Gjelder også for tilnærma uten å forutsette normalfordelt (stor $n_1$ og $n_2$)\\
					\textbf{For ukjent men lik varians} $\sigma_1^2=\sigma_2^2=\sigma_p^2$ estimeres med $S^2_p$:\\
					$T=\frac{(\bar X_1 - \bar X_2)-(\mu_1-\mu_2)}{S_p\sqrt{\frac{1}{n_1}+\frac{1}{n_2}}}\sim T_{n_1+n_2-2}$\\
					\textbf{For ukjent og ulik varians} $\sigma_1^2\neq\sigma_2^2$, estimeres med $S_1^2,S_2^2$\\
					$T=\frac{(\bar X_1 - \bar X_2)-(\mu_1-\mu_2)}{\sqrt{\frac{S_1^2}{n_1}+\frac{S_2^2}{n_2}}}\sim T_{\nu}\quad \nu=\frac{\left(\frac{S_1^2}{n_1} + \frac{S_2^2}{n_2} \right)^2}{\frac{\left(\frac{S_1^2}{n_1}\right)^2}{n_1-1} + \frac{\left( \frac{S_2^2}{n_2} \right)^2}{n_2-1}}$\\
					$F$: For to uavhengige tilfeldige utvalg $n_1,n_2$ (normalfordelt) med forventning $\mu_1,\mu_2$ og varianser $\sigma_1^2,\sigma_2^2$ har vi at: \\
					$F=\frac{\frac{S_1^2}{n_1}}{\frac{S_2^2}{n_2}}=\frac{S_1^2\sigma_2^2}{S_2^2\sigma_1^2}\sim F_{n_1-1,n_2-1}$\\
					$\hat p_1 - \hat p_2$: med to tilfeldig utvalg $n_1,n_2$ (binomisk) med sannsynligheter $p_1,p_2$ har vi tilnærma at (om $n_1,n_2$ stor nok):\\
					$Z=\frac{(\hat p_1 -\hat p_2)-(p_1-p_2)}{\sqrt{\frac{p_1(1-p_1)}{n_1}+\frac{p_2(1-p_2)}{n_2}}}\sim N(0,1)$\\
					$\bar D$: for to parvise tilfeldige utvalg av størrelse $n$ der differansene er normalfordelte med forventning $\mu_D$ og varians $\sigma_D^2$ og der variansen estimeres ved $S_D^2$ fra utvalget har vi at:\\
					$T=\frac{\bar D - \mu_D}{\frac{S_D}{\sqrt n}}\sim t_{n-1}$
				}
			\end{block}
		\end{column}
		\begin{column}{.25\linewidth}
			\begin{block}{\center Intervallestimering}
				\tema{Konfidensintervall for $\mu$ (KI)}
				{KI for $\mu$ med kjent $\sigma^2$: \\
					$P\left( -z_{\frac{\alpha}{2}} < \frac{\bar X-\mu}{\frac{\sigma}{\sqrt{n}}} < z_{\frac{\alpha}{2}} \right) = 1-\alpha$\\
					$\Rightarrow \mu = \bar X \pm z_{\frac{\alpha}{2}} \cdot\frac{\sigma}{\sqrt{n}}$ \\
					KI for $\mu$ med $S^2$ som estimator for ukjent $\sigma^2$
					med $n-1$ frihetsgrader:
					$P\left( -t_{\frac{\alpha}{2}} < \frac{\bar X-\mu}{\frac{s}{\sqrt{n}}} < t_{\frac{\alpha}{2}} \right) = 1-\alpha $\\ $\Rightarrow \mu = \bar X \pm t_{\frac{\alpha}{2}} \cdot\frac{s}{\sqrt{n}}$ \\
				}
				\tema{Konfidensintervall for $\sigma^2$ med $n-1$ (KI)}
				{$P\left( \mathcal{X}^2_{1-\frac{\alpha}{2}} <\frac{(n-1)S^2}{\sigma^2} < \mathcal{X}^2_{\frac{\alpha}{2}} \right)$ \\
					$\Rightarrow \sigma^2 \in \left[ \frac{(n-1)S^2}{\mathcal{X}^2_{\frac{\alpha}{2}}}, \frac{(n-1)S^2}{\mathcal{X}^2_{1-\frac{\alpha}{2}}} \right]$
				}
				\tema{Konfidensintervall for $p$}
				{KI for $p$ med utvalg av størrelse $n$ fra et binomisk forsøk med sannsynlighet $p$ (OBS: $n>30$ /stor nok)
					$P\left( -z_{\frac{\alpha}{2}} < \frac{\hat p - p}{\sqrt{\frac{p(1-p)}{n}}} < z_{\frac{\alpha}{2}} \right) = 1 - \alpha$ \\ 
					$\Rightarrow p = \hat p \pm z_\frac{\alpha}{2} \sqrt{\frac{p(1-p)}{n}}$
					Erstatt $p$ i utrykket med $\hat p$ for å kunne regne ut\\
				}
				\tema{Konfidensintervall for $\mu_1 - \mu_2$}
				{For to uavhengige tilfelig utvalgte utvalg av størrelser $n_1$ og $n_2$ fra normalfordelte populasjoner med forventning $\mu_1, \mu_2$ og vareanser $\sigma_1^2, \sigma_2^2$: \\
					For \textbf{kjente} vareanser $\sigma_1^2, \sigma_2^2$: \\
					$\mu_1-\mu_2=(\bar x_1 - \bar x_2) \pm z_\frac{\alpha}{2}\sqrt{\frac{\sigma_1^2}{n_1}+\frac{\sigma_2^2}{n_2}}=1-\alpha$\\
					For \textbf{ukjente men like} vareanser $\sigma_1^2=\sigma_2^2$ med antall frihetsgrader $\nu = n_1 + n_2 - 2$:\\
					$\mu_1-\mu_2=(\bar x_1 - \bar x_2) \pm s_p\sqrt{\frac{1}{n_1}+\frac{1}{n_2}}=1-\alpha$\\
					For \textbf{ukjente men ulike} vareanser $\sigma_1^2\neq \sigma_2^2$ der antall frihetsgrader $\nu$ er gitt fra utvalgsfordelinga: \\
					$\mu_1-\mu_2=(\bar x_1 - \bar x_2) \pm t_\frac{\alpha}{2}\sqrt{\frac{s_1^2}{n_1}+\frac{s_2^2}{n_2}}=1-\alpha$\\
				}
				\tema{Konfidensintervall for $\frac{\sigma_1^2}{\sigma_2^2}$}
				{Normalfordelt, med $\mu_1,\mu_2$ og $\sigma_1^2,\sigma_2^2$:\\
					$P\left( f_{1-\frac{\alpha}{2},\nu_1,\nu_2} < \frac{S_1^2\sigma_2^2}{S_2^2\sigma_1^2} <  f_{\frac{\alpha}{2},\nu_1,\nu_2}\right)$\\
					$\Rightarrow \frac{\sigma_1^2}{\sigma_2^2} \in \left[ \frac{S_1^2}{S_2^2}\frac{1}{f_{\frac{\alpha}{2},\nu_1,\nu_2}}, \frac{S_1^2}{S_2^2}f_{\frac{\alpha}{2},\nu_1,\nu_2} \right] $
				}
				\tema{Konfidensintervall for $p_1-p_2$}
				{Uavhengig tilfeldig utvalg $n_1,n_2$ fra binomisk forsøk med sannsynlighet $p_1,p_2$ har vi (om $n_1,n_2$ store nok):\\
					$(\hat p_1 - \hat p_2) \pm z_\frac{\alpha}{2} \sqrt{\frac{\hat p_1 (1 - \hat p_1)}{n_1} + \frac{\hat p_2 (1 - \hat p_2)}{n_2}}=1-\alpha$
				}
				\tema{Konfidensintervall for $\mu_D$}
				{For to parvise tilfeldige utvalg, størrelse $n$ med normalfordelte differanse med forventning $\mu_D$ og varians $\sigma^2_D$ (Variansen estimeres med $S_D^2$) \\
					$\bar d \pm t_\frac{\alpha}{2}\frac{S_D}{\sqrt{n}}=1-\alpha$\\
				}
				\tema{Lengden av KI}
				{For en normalfordelt estimator med kjent varians $\sigma^2$ kan vi regne ut hva lengden vil bli for et gitt valg av $n$ og $\alpha$ \\
					$L=2\cdot z_\frac{\alpha}{2}\cdot SE(\hat\Theta)$, to typiske tilfeller:
					\begin{enumerate}
						\item Konstruere KI for $\mu$ basert på estimatoren: $\bar X$ blir lengda:\\
							$L=2\cdot z_\frac{\alpha}{2}\cdot \frac{\sigma}{\sqrt{n}}\Rightarrow n=\left( \frac{2z_\frac{\alpha}{2}\sigma}{L} \right)^2$\\
							Når $\sigma^2$ må estiemres kan vi finne approx forventa lengde\\
						\item Når vi konstruerer KI for $p$ basert på andelsestimatoren $\hat p$ ($l \approx L$):\\
							$L=2\cdot z_\frac{\alpha}{2}\sqrt{\frac{\hat p (1 - \hat p)}{l}}\leq \frac{z_\frac{\alpha}{2}}{\sqrt{n}}\Rightarrow n=\left( \frac{2z_\frac{\alpha}{2}\sqrt{\hat p (1 - \hat p)}}{l} \right)^2$

					\end{enumerate}
				}
				\subsect{Estimeringsfeil}
				\body{Når fordelinga til estimatoren er kjent (utvalgsfordelinga) kan vi regne ut hvor stor feil vi gjør i estimeringa. Vil ofte ha en viss sannsynlighet for at feilen ikke skal overskride en verdi $e$\\
					$P(|\hat\Theta-\Theta|<e=1-\alpha)$ (vanlige tilfeller)\\
					$\bar X$ (normalfordelt):\\
					$P(|\bar X - \mu|<e=1-\alpha) \Rightarrow n=\left( \frac{z_\frac{\alpha}{2}\sigma}{e}\right)^2\Rightarrow e=z_\frac{\alpha}{2}\frac{\sigma}{\sqrt n}$\\
					$\hat p$ (binomisk):\\
					$P(|\hat p - p| < e=1-\alpha)\Rightarrow n\approx \left(\frac{z_\frac{\alpha}{2}\sqrt{\hat p {1-\hat p}}}{e}\right)^2\Rightarrow e\approx z_\frac{\alpha}{2}\sqrt{\frac{\hat p(1 - \hat p)}{n}}$}
				\subsect{Noen regneregler}
				\body{
					$(u\cdot v)' = u'v + u\cdot v'$\\
					$\left( \frac{u}{v}\right) = \frac{u'v - u\cdot v'}{v^2}$\\
					$g(f(x))'=g'(f(x))\cdot f'(x)$\\
					$\ln(x)'=\frac{1}{x}$\\
					$\int \ln (x) \, dx=x\ln (x) -x + C$ \\
					$u$ er et utrykk med $x$\\
					$\int a^u \, dx = \frac{a^u}{u'\cdot \ln(a)} + C$\\
					$\int \frac{1}{u} \,dx = \frac{\ln(u)}{u'} + C$ \\
					$\int u \cdot v' \, dx = u\cdot v-\int u'v \, dx$
				}
				\subsect{Kombinatorikk}
				\body{\textbf{Multiplikasjonsregelen}\\
					Forsøk i $k$ etapper, med $m_1,m_2 \cdots,m_k$ mulige utfall i etappene. Totalt antall utfall: $\prod_{i=1}^k m_i$\\
					\textbf{Potensregelen}\\
					$n$ merka enheter, velger $k$ med tilbakelegging. Antall ordna utfall: $n^k$\\
					\textbf{Permutasjonsregelen (nPr)}\\
					$n$ merka enheter, velger $k$ uten tilbakelegging, antall ordna utfall: $_nP_r=\frac{n!}{(n-k)!}$\\
					\textbf{Kombinasjonsregelen}\\
					$n$ merka enheter, velger $k$ uten tilbakelegging, antall ikke-ordna utfall: $_nC_r=\binom{n}{k}=\frac{n!}{(n-k)!k!}$\\}
			\end{block}
		\end{column}
		\begin{column}{.25\linewidth}
			\begin{block}{}
				\center\normalsize Prediksjonintervall\\
				{\footnotesize\raggedright
					Et intervall som sier noe om neste verdi $X_0$ i en normalfordelt populasjon: $X\sim N(\mu,\sigma)$\\
					\textbf{Kjent $\mu$ og $\sigma$} (spredningsintervall)\\
					$P\left( \mu - z_\frac{\alpha}{2} \cdot \sigma < X_0 < \mu + z_\frac{\alpha}{2} \cdot \sigma \right)=1-\alpha$\\
					$\Rightarrow X_0=\mu \pm z_\frac{\alpha}{2}\cdot \sigma$\\
					\textbf{Ukjent $\mu$, kjent $\sigma$} \\
					$P\left( \bar X - z_\frac{\alpha}{2} \cdot \sigma \sqrt{1 + \frac{1}{n}} < X_0 < \bar X + z_\frac{\alpha}{2} \cdot \sigma \sqrt{1 + \frac{1}{n}} \right)=1-\alpha$\\
					$\Rightarrow X_0=\bar x \pm z_\frac{\alpha}{2}\cdot \sigma \sqrt{1 + \frac{1}{n}}$\\
					\textbf{Ukjent $\mu, \sigma$} frihetsgrad $\nu=n-1$\\
					$X_0=\bar x \pm z_\frac{\alpha}{2} \cdot S\sqrt{1 + \frac{1}{n}}$ \\
					\textbf{Formel for $S_P^2$} (Gjelder for alt på arket) \\
					$S_P^2=\frac{S_1^2(n_1-1)+S_2^2(n_2-1)}{n_1 + n_2 - 2}$
				}
			\end{block}
			%\end{column}
			%\begin{column}{.25\linewidth}
			\begin{block}{\center\normalsize Fordelinger}
				{\footnotesize\raggedright
					\textbf{Binomisk fordeling}\\
					1. $n$ uavhengige delforsøk\\
					2. To utfall: suksess/ikke suksess\\
					3. Samme sannsynlighet for $p=P(a)$ (suksess) i alle delforsøk\\
					\textbf{Hypergeometrisk fordeling}\\
					1. Populasjon med $N$ elementer\\
					2. $k$ av disse regnes som ``suksess'', $N-k$ som ``fiasko''\\
					3. Trekker $n$ elementer uten tilbakelegging\\
					Sannsynligheten $p$ endrer seg mellom hvert delforsøk.
					\textbf{Negativ binomisk fordeling}\\
					Antall forsøk du må gjøre for at hendelsen $A$ (suksess) skal intreffe $k$ ganger\\
					\textbf{Geometrisk fordeling}\\
					Antall forsøk du må gjøre for at hendelsen $A$ (suksess) skal intreffeførste gang\\
					\textbf{Poisson-fordeling}\\
					$\mu = \lambda t$, $\sigma^2=Var(X)=\lambda t$\par
					$f(x)=\frac{\mu^x}{x!}e^{-x}$\par
					1. Antallet av $A$ disjunkte tidsintervall er uavhengige\\
					2. Forventa antall av $A$ er konstant list $\lambda$ (raten) per tidsenhet\\
					3. Kan ikke få to forekomster samtidig\\
					\textbf{Gammafordeling}\\
					En kontinuerlig vareabel $X$ er gammafordelt med parameter $\alpha > 0$ og $\beta > 0$ dersom tetthetsfunksjonen er gitt ved (se blå tabell). Ventetida til hendelse nummer $k$ i en Poisson-prosess vil være gammafordelt med $\alpha=k$ og $\beta=\frac{1}{\lambda}$\\
					\textbf{Eksponensialfordeling}\\
					Ventetida til første hendelse (og mellom etterfølgende handelser) i en Poisson-prosess følger en eksponensialfordeling. En kontinuerlig variabel $X$ har eksponentialfordeling med parameter $\beta > 0$ dersom tetthetsfunksjonen er gitt ved (se tabellbok: $f(x)$). Eksponensialfordelinga er en variant av gammafordelinga med $\alpha=1$\\
				}
				\center\normalsize Stokastisk variabel\\
				{\footnotesize\raggedright
					\textbf{En variabel}\\
					$\mu=E(X)=\begin{cases}
						\sum_x x\cdot f(x), &\text{deskret}\\
						\int_{-\infty}^\infty x\cdot f(x) dx, &\text{kontinuerlig}
					\end{cases}$\\
					$\sigma^2=Var(X)=E(X^2)-E(X)^2=\begin{cases}
						\sum_x (x-\mu)^2f(x), &\text{deskret}\\
						\int_{-\infty}^\infty (x-\mu)^2\cdot f(x) dx, &\text{kontinuerlig}
					\end{cases}$\\
					$F(x)=\int_0^xf(t)\,dt$\\
					\textbf{Funksjoner av stokastiske variabler}\\
					$\mu_{g(X)}=E(g(X))=\begin{cases}
						\sum_x g(x)\cdot f(x), &\text{deskret}\\
						\int_{-\infty}^\infty g(x) \cdot f(x) dx, &\text{kontinuerlig}
					\end{cases}$\\
					$\sigma^2_{g(X)}=Var(g(X))=\begin{cases}
						\sum_x (g(x)-\mu_{g(X)})^2\cdot f(x), &\text{deskret}\\
						\int_{-\infty}^\infty (g(x)-\mu_{g(X)})^2 \cdot f(x) dx, &\text{kontinuerlig}
					\end{cases}$\\
					$\mu_{g(X,Y)}=E(g(X,Y))=\begin{cases}
						\sum_x\sum_x g(x,y) f(x,y), &\text{deskret}\\
						\int_{-\infty}^\infty\int_{-\infty}^\infty g(x,y) f(x,y)\,dx\,dy &\text{kontinuerlig}
					\end{cases}$\\
					\textbf{Simultanfordeling for to variabler}\\
					$P(X,Y)=\begin{cases}
						\sum\sum f(x,y),&\text{deskret}\\
						\int\int f(x,y)\,dx\,dy, &\text{kontinuerlig}
					\end{cases}$\\
					\textbf{Marginale fordelinger} (to variabler)\\
					$g(x)=\begin{cases}
						\sum_y f(x,y)\\
						\int_{-\infty}^\infty f(x,y)\,dy
					\end{cases}$\\
					$h(y)=\begin{cases}
						\sum_x f(x,y)\\
						\int_{-\infty}^\infty f(x,y)\,dx
					\end{cases}$\\
					\textbf{Betinga fordeling} (for Y gitt X)\\
					$f(y|x)=\frac{f(x,y)}{g(x)},\quad g(x)>0$\\
					\textbf{Kovarians}\\
					$\sigma_{XY}=Cov(X,Y)=\begin{cases}
						\sum_x\sum_y(x-\mu_X)(y-\mu_Y)f(x,y)\\
						\int_{-\infty}^\infty\int_{-\infty}^\infty(x-\mu_X)(y-\mu_Y)f(x,y) \, dx \, dy
					\end{cases}$\\
					Korrelasjon:\\
					$\rho_{XY}=\frac{Cov(X,Y)}{\sqrt{Var(x)\cdot Var(y)}}, \qquad -1 < \rho_{XY} < 1$\\
				}
				\center\normalsize Sannsynlighetsregler\\
				{\footnotesize\raggedright
					\textbf{Addisjonsregelen}: $P(A\cup B)=P(A)+P(B)-P(A \cap B)$\\
					\textbf{Multiplikasjonsregelen}: $P(A\cap B)=P(A)\cdot P(B|A)=P(B)\cdot P(A|B)$\\
					\textbf{Betinga sannsynlighet}: $P(A|B)=\frac{P(A\cap B)}{P(B)}$\\
					\textbf{Total sannsynlighet}: $P(A)=\sum_{i=1}^n P(B_i\cap A)=\sum_{i=1}^kP(B_i)\cdot P(A|B_i)$\\
					\textbf{Bayes setning}: $P(B|A)=\frac{P(B)\cdot P(A|B)}{P(A)}$\\
					\textbf{Tilfeller ved uavhengighet:}\\
					$P(A\cap B)=P(A)\cdot P(B)$\\
					$P(A|B)=A(A)$\\
				}
			\end{block}
		\end{column}
		\begin{column}{.25\linewidth}
			\begin{block}{\center\normalsize Testing}
				{\footnotesize\raggedright
					Lager tester som tar stilling til påstander. Har da: nullhypotesen ($H_0$) og en utfordrende hypotese ($H_1$). Vi bruker testobservatorer for å kunne si nø om $H_0$ er sann eller ikke.\\
					\textbf{Normalfordelt estimator:}\\
					Dersom vi har en normalfordelt forventningsrett estimator $\hat\Theta$ for parameter $\Theta$: $\hat\Theta\sim N(\Theta,SE(\hat\Theta))$\\
					Kan vi konstruere testobservator $Z=\frac{\hat\Theta-\Theta_0}{SE(\hat\Theta_0)}$\\
					(Måler hvor mange standardavvik $\hat\Theta$ er fra $\Theta_0$.)\\
					-Om $H_0$ er sann vil $Z$ bli standardnormalfordelt\\
					-Om $H_1$ er sann vil vi forvente at $\hat\Theta$ blir større enn $\Theta_0$, og dermed at $Z$ blir stor.\\
					Forkastningsområdet velges slik at med sannsynlighet $\alpha$ for å få en så stor verdi, dersom $H_0$ er sann. Kritisk verdi blir da $z_\alpha$ om $Z>z_\alpha$\\
					\textbf{P-verdi}:\\
					Som et alternativ fil å finne et forkastningsområde for $Z$ kan man finne en $p-verdi$ som er det minste signifikansnivået som ville forkastet nullhypotesen.\\
					\textbf{Feil av type I/II}:\\
					P(Feil av type I) = P(Forkaste en rett $H_0$)=$\alpha$\\
					P(Feil av type II) = P(ikke forkaste gal $H_0$)=$\beta$\\
					Det ideelle er å få $\alpha$ og $\beta$ minst mulig.\\
					$1-\beta$ kalles for styrken til en test og sier hvor sannsynlig det er å forkaste $H_0$ som funskjon av den sanne parameterverdien. Ved å finne denne for ulike verdier av parameteren akn vi skissere en styrkefunksjon.\\
					\textbf{Ulike hypoteseoppsett}:\\
					Om vi har $H_0\Rightarrow \hat\Theta=\Theta_0$ har vi tre tester. Venstresidig, tosidig og høyresidig. $H_1$ blir da respektivt:\\
					$\hat\Theta<\Theta_0\qquad\hat\Theta\neq\Theta_0\qquad\hat\Theta>\Theta_0$\\
					Hvis ikke en retning er indikert er en tosidig test å foretrekke.\\
					\textbf{Noen vanlige tester, standardsituasjoner}\\
					$\mu$: fra normalfordelt utvalg av størrelse $n$, kjent $\sigma^2$. Testobservator blir da (for alle testene): $Z=\frac{\bar X - \mu_0}{\frac{\sigma}{\sqrt n}}$\\
					Som forkaster når (VS Tosidig HS)\\
					$z<-z_\alpha\qquad |z|>z_\frac{\alpha}{2}\qquad z>z_\alpha$\\
					$\mu$: fra normalfordelt utvalg av størrelse $n$, ukjent $\sigma^2$, bruker $S^2$. Testobservator vil bli $t-fordelt$ med $n-1$ firhetsgrader\\
					Testobservator blir da (for alle testene): $t=\frac{\bar X-\mu_0}{\frac{S}{\sqrt n}}$
					Som forkaster når (VS Tosidig HS)\\
					$t<-t_\alpha\qquad |t|>t_\frac{\alpha}{2}\qquad t>t_\alpha$\\
					$\sigma^2$: fra normalfordelt utvalg av størrelse $n$. Testobservator vil bli $kjikvadratfordelt\,(\chi^2)$ med $n-1$ frihetsgrader.\\
					Testobservator for alle testene: $\chi^2=\frac{(n-1)S^2}{\sigma_0^2}$\\
					Som forkastes når: (VS Tosidig HS)\\
					$\chi^2<-\chi^2_{1-\alpha}\qquad\chi^2<\chi^2_{1-\frac{\alpha}{2}}\vee\chi^2>\chi^2_\frac{\alpha}{2}$\\$ \qquad\chi^2>\chi^2_\alpha$\\
					$p$ for et tilfeldig utvalg av størrelse $n$ fra et binomisk forsøk med sannsynlighet $p$. Hvis $n$ stor nok ($n \geq 30$) kan tilnærmes normalfordelt.\\
					Testobservator blir da: $Z=\frac{\hat p - p_0}{\sqrt{\frac{p_0(1-p_0)}{n}}}$, grenser = grensene for $\mu$\\
					\textbf{Hvis $n<30$} må man bruke binomisk fordeling, vanlig testobservator er da $X=$ antall suksesser.\\
					Vanlig å bruke \textbf{p-verdi} for dette tilfellet.\\
					$\mu_1-\mu_2$: For to uavhengige utvalg av størrelser $n_1,n_2$ (normalfordelt) med forventninger $\mu_1,\mu_2$ og varianser $\sigma^2_1,\sigma_2^2$. Dette gir $H_0$: $\mu_1-\mu_2=d_0$ (en vanlig test er $d_0=0$)\\
					\textbf{Kjent varians} $\sigma_1^2,\sigma_2^2$: Testobservatoren blir da\\
					$Z=\frac{(\bar X_1-\bar X_2)-d_0}{\sqrt{\frac{\sigma_1^2}{n_1}+\frac{\sigma_2^2}{n_2}}}$ forkasningsgrensene blir som en vanlig Z test ($\mu$ med ukjent $\sigma^2$)\\
					\textbf{ukjent og lik varians} $\sigma_1^2=\sigma_2^2=\sigma^2$:\\
					Testobservatoren blir $t-fordelt$ med frihetsgrad $n_1 + n_2-2$:
					$T=\frac{(\bar X_1 - \bar X_2) - d_0}{S_p\sqrt{\frac{1}{n_1}+\frac{1}{n_2}}}$\\
					Forkastningsgrensene blir som en vanlig $T$ test ($\mu$ med ukjent $\sigma^2$)\\
					\textbf{ukjent og ulik varians} $\sigma_1^2\neq\sigma_2^2$:\\
					Testobservatoren blir \textit{t-fordelt} med frihetsgrad gitt fra utvalgsfordelinga. Testobservator blir da:
					$T=\frac{(\bar X_1 - \bar X_2) - d_0}{\sqrt{\frac{S_1^2}{n_1}+\frac{S_2^2}{n_2}}}$\\
					$\frac{\sigma_1^2}{\sigma_2^2}$: for to uavhengige utvalg av størrelser $n_1,n_2$ (normalfordelt) med forventninger $\mu_1,\mu_2$ og varianser $\sigma_1^2,\sigma_2^2$. Testobservatoren blir da \textit{f-fordelt} med firhetsgrader $n_1-1$ og $n_2-1$. Testobservatoren blir da (for alle testene): $F=\frac{S_1^2}{S_2^2}$ som forkastes når \\(VS Tosidig HS)\\
					$F<\frac{1}{F_{\alpha,\nu_1,\nu_2}} \qquad F_{\frac{\alpha}{2},\nu_1,\nu_2} < F < \frac{1}{F_{\frac{\alpha}{2},\nu_1,\nu_2}} $\\$ F > F_{\alpha,\nu_1,\nu_2}$\\
					$p_1-p_2$: For to uavhengige tilfeldige utvalg av størrelse $n_1,n_2$ (binomisk) med sannsynligheter $p_1,p_2$. Får $H_0:p_1-p_2=d_0$. Testobservator blir\\
					$\frac{(\hat p_1 - \hat p_2)-d_0}{\sqrt{\frac{\hat p_1(1-\hat p_1)}{n_1}+\frac{\hat p_2 (1-\hat p_2)}{n_2}}}$ \\Hvis $d_0=0$:$\frac{\hat p_1 - \hat p_2}{\sqrt{p(1-p)}\left(\frac{1}{n_1}+\frac{1}{n_2}\right)}$ Standardnormalfordelt ved stor nok $n_1,n_2$. Forkastningsgrense blir som en vanlig Z test ($\mu$ med kjent $\sigma^2$)\\
					$\mu_D$: Differansene er normalfordelt, variansen $\sigma_D^2$ må estimeres $S_D^2$. Får hypotesen $H_0:\mu_D=d_0$\\
					Testobservatoren er \textit{t-fordelt} med $n-1$ firhetsgrader. Testobservatoren blir da: $T=\frac{\bar D - d_0}{\frac{S_D}{\sqrt n}}$. Forkastningsområdet blir deretter som en vanlig T-test.
				}
				\center\normalsize Styrkefunksjon\\
				{\footnotesize\raggedright
					Ved å finne sannsynligheten for å forkaste $H_0$ for ulike verdier av parameteren, $\Theta$, finner vi styrkefunksjonen:\\
					$1-\beta(\Theta)=P$(Forkaste $H_0$, sann verdi = $\Theta$)
					$1-\beta(\mu)=1-\Phi\left(z_\alpha-\frac{\mu-\mu_0}{\frac{\sigma}{\sqrt n}}\right) \qquad n=\frac{(z_\alpha-z_\beta)^2\sigma^2}{(\mu-\mu_0)}$
				}
			\end{block}
		\end{column}
	\end{columns}
\end{frame}
\begin{frame}{} 
	\begin{columns}[t]
		\begin{column}{.33\linewidth}
			\begin{block}{}
				\subsect{Forenklinger for lineærkombinasjoner}
				\body{\footnotesize\raggedright
					$Var(aX+bY)=a^2Var(X)+b^2Var(Y)+2ab\cdot Cov(X,Y)$\\
					Merk: hvis X og Y er uavhengige er $Cov(X,Y)=0$\\
					$E(aX+b)=aE(X)+b$\\
					$E(\chi^2_{n-1})=n-1$\\
					$E(T_\nu)=0$\\
					$Var(T_\nu)=\frac \nu {\nu - 2}, \qquad \nu \geq 3$\\
					\textbf{Tsjebysjeffs teorem}: Sannsynligheten for at verdien på en variabel $X$ ligger innafor $k$ avstander fra forventningsverdien er minst $1-\frac{1}{k^2}$\\
					\textbf{Sannsynlighetsmaksimeringsfunksjonen}:\\
					$L(\Theta)=\prod_{i=1}^n f(x_i,\Theta)$\\
					Løses ved å:\\
					1. Ta logaritmen av funksjonen $\ln L$\\\vspace{-2mm}
					\maths{&\ln\left(\frac{a}{b}\right)=\ln(a)-\ln(b),&\ln(ab)=\ln(a)+\ln(b),& &\ln(a^b)=b\ln (a)}\vspace{-3mm}
					\maths{&\ln\left(\prod a\right)=\sum\ln(a)}
					2. Deriver utrykker med hensyn på variabelen $\Theta$ oftest $\beta$, $\frac{\partial\ln L}{\partial \Theta}$\\
					3. Sett utrykket lik $0$ og løs med hensyn på variabelen.$\frac{\partial\ln L}{\partial \Theta}=0$\\
					\textbf{Tilfeldig utvalg}:\\
					Når et utvalg er tilfeldig, kan vi se på det som en mengde identiske og uavhengige observasjoner.\\
					\textbf{QQ-plot}:\\
					- Plotter utvalgskvantiler (Observasjonene ordna etter størrelse) mot teoretiske kvantiler (``Ideelle observasjoner'') fra aktuell fordeling\\
					- Teoretiske kvantiler er gitt ver invers kumulativ fordeling ``jevnt spredte'' sannsynligheter mellom 0 og 1\\
					- Om antatt fordeling stemmer skal plotter gi en tilnerma rett linje $x=y$\\
				}\vspace{-3mm}
				\subsect{Enkel linær regresjon}
				\body{Har observasjonspar $(x_1,y_1),\ldots,(x_n,y_n)$.\\}
				\tema{Regresjonsmodellen}
				{\vspace{-3mm}\maths{&Y_i=\beta_0+\beta_1x_i+\epsilon_i, &i=1,\ldots,n}
					Forutsetningene er at
					\maths{&E(\epsilon_i)=0,&Var(\epsilon_i)=\sigma^2,& &i=1,\ldots,n}
					og $\epsilon_1,\ldots,\epsilon_n$ er inbyrdes uavhengige. Dette medfører at
					\maths{&E(Y_i)=\mu_{Y|x_i}=\beta_0+\beta_1x_i,&Var(Y_i)=\sigma_{Y|x_i}^2=\sigma^2,& &i=1,\ldots,n}
					der $Y_1,\ldots,Y_n$ også er innbyrdes uavhengige.}
				\tema{MKM}
				{Bruker miste kvadrat-metoden til å estimere $\beta_0$ og $\beta_1$ fra data for å få ei estimert (tilpassa) regresjonslinje for forventningslinja $\beta_0+\beta_1x$:
					\maths{\hat y=b_0+b_1x}
					MKM baserer seg på å minimere
					\maths{SSE=\sum_{i=1}^n(y_i-\hat y_i)^2=\sum_{i=1}^n(y_i-b_0-b_1x_i)^2}
					Gir estimatorer
					\maths{B_1=\frac{\sum_{i=1}^n(x_i-\bar x)(Y_i-\overline{Y})}{\sum_{i=1}^n(x_i-\overline{x})^2}=\frac{\sum_{i=1}^n(x_i-\bar x)Y_i}{\sum_{i=1}^n(x_i-\overline{x})^2}, &B_0=\overline{Y}-B_1\overline{x}}
					der
					\maths{&E(B_1)=\beta_1, &Var(B_1)=\frac{\sigma^2}{\sum_{i=1}^n(x_i-\overline x)^2}\\
						&E(B_0)=\beta_0, &Var(B_0)=\frac{\sigma^2\sum_{i=1}^nx_i^2}{n\sum_{i=1}^n(x_i-\overline{x})^2}}
					I tillegg har vi (forventningsrett) estimator for $\sigma^2$:
					\maths{S^2=\frac{SSE}{n-2}=\frac{\sum_{i=1}^n(Y_i-\hat Y_i)^2}{n-2}=\frac{\sum_{i=1}^n(Y_i-B_0-B_1x_i)^2}{n-2}}}
				\tema{Parameterinferens}
				{Om vi anntar at feilledda er normalfordelte
					\maths{\epsilon_i\sim N(0,\sigma)}
					vil
					\maths{&B_1\sim N\left(\beta_1,\frac{\sigma}{\sqrt{\sum_{n=1}^n(x_i-\ol x)^2}}\right), &B_0\sim N\left(\beta_0,\sqrt{\frac{\sum_{i=1}^nx_i^2}{n\cdot\sum_{i=1}^n(x_i-\ol x)^2}}\right)}
					og KI og tester kan konstrueres ved å ta utgangspunkt i at
					\maths{V=\frac{(n-2)S^2}{\sigma^2}}
					er kikvadratfordelt med $n-2$ frihetsgrader, og dermed at
					\maths{&T=\frac{B_1-\beta_1}{\hat{SE}(B_1)}=\frac{B_1-\beta_1}{\frac{S}{\sqrt{\sum_{i=1}^n(x_i-\ol x)^2}}}, &T=\frac{B_0-\beta_0}{\hat{SE}(B_0)}=\frac{B_0-\beta_0}{S\sqrt{\frac{\sum_{i=1}^nx_i^2}{n\cdot\sum_{i=1}^n(x_i-\ol x)^2}}}}
					vil være t-fordelte med $n-2$ frihetsgrader. Dette gir oss da KI og tester ofr $\beta_1,\beta_0,\sigma$}
				\tema{Inferens for $\mu_{Y|x_0}$}
				{Dersom $x$ er en gitt verdi, $x_0$, vil en estimator for forventa verdi av responsen bli
					\maths{\hat Y_0=B_0+B_1x_0}
					Fordelinga for $\hat Y_0$ blir:
					\maths{\hat Y_0 \sim N\left(\mu_{Y|x_0},\sigma\sqrt{\frac{1}{n}+\frac{(x_0-\ol x)^2}{\sum_{i=1}^n(x_i-\ol x)^2}}\right)}
					Og da kan ta utgangspunkt i at
					\maths{T=\frac{\hat Y_0-\mu_{Y|x_0}}{S\sqrt{\frac{1}{n}+\frac{(x_0-\ol)^2}{\sum_{i=1}^n(x_1-\ol x)^2}}}}
					er t-fordelt med $n-2$ frihetsgrader til for eksempel å lage $100(1-\alpha)\%$-konfidensintervall for $\mu_{Y|x_0}$ (forventa verdi av responsen $Y$ når $x=x_0$):
					\maths{\left[\hat y_0 - t_{\frac{\alpha}{2}}s\sqrt{\frac{1}{n}+\frac{(x_0-\ol x)^2}{\sum_{i=1}^n(x_i - \ol x)^2}},\hat y_0 + t_{\frac{\alpha}{2}}s\sqrt{\frac{1}{n}+\frac{(x_0-\ol x)^2}{\sum_{i=1}^n(x_i - \ol x)^2}}\right]}.}
			\end{block}
		\end{column}
		\begin{column}{.33\linewidth}
				\tema{PI for $Y_0$}
				{Ønsker ofte intervall som med sannsynligheten $100(1-\alpha)\%$ vil inneholde verdien av en ny observasjon $Y_0$ (når $x=x_0$), altså et \textit{prediksjonsintervall}. Tar utgangspunkt i forskjellen mellom estimatoren for forventa respons $\hat Y_0$ og verdien av en ny observasjon, $Y_0$
					\maths{\hat Y_0 - Y_0 \sim N\left(0,\sigma\sqrt{1+\frac{1}{n}+\frac{(x_0-\ol x)^2}{\sum_{i=1}^n(x_i-\ol x)^2}}\right)}
					Da kan vi bruke at
					\maths{T=\frac{\hat Y_0-Y_0}{S\sqrt{1+\frac{1}{n}+\frac{(x_0-\ol x)^2}{\sum_{i=1}^n(x_i-\ol x)^2}}}}
					er t-fordelt med $n-2$ frihetsgrader.\\
					Et $100(1-\alpha)\%$-PI for responsen $Y_0$ vil da finnes ved
					\maths{\left[\hat y_0-t_\frac{\alpha}{2}s\sqrt{1+\frac{1}{n}+\frac{(x_0-\ol x)^2}{\sum_{i=1}^n(x_i-\ol x)^2}},\hat y_0+t_\frac{\alpha}{2}s\sqrt{1+\frac{1}{n}+\frac{(x_0-\ol x)^2}{\sum_{i=1}^n(x_i-\ol x)^2}}\right]}.
				}
				\tema{Betydning av lineær modell}
				{\maths{\text{Total variasjon}&=\text{Forklart variasjon}&&+\text{Restvariasjon}\\
						\sum_{i=1}^n(y_i-\ol y)^2&=\sum_{i=1}^n(\hat y_i-\ol y)^2 &&+ \sum_{i=1}^n (y_i-\hat y_i)^2\\
						SST&=SSR&&+SSE}
					\textbf{forklaringsgraden} er definert ved
					\maths{R^2&=\frac{\text{Forklart variasjon}}{\text{Total variasjon}}=\frac{SSR}{SST}=\frac{SST-SSE}{SST}=1-\frac{SSE}{SST}, &0 \leq R^2 \leq 1}
					Dette er et vanlig mål på hvor stor andel av variasjonen i responsen som kan tilskrives den lineære sammenhengen $\mu_{Y|x}=\beta_0+\beta_1 x$ (Resten tilskrives tilfeldige feil, $\epsilon$)}
				\begin{block}{Løsningsforslag}
				\oppgave{Vi anntar at konsentrasjonen i en prøve $Y_i$, av volum $l_i$ liter har følgende fordeling:
					\maths{Y_i\sim N\left(\mu,\frac{\sigma}{\sqrt l_i}\right)}
					Nå er det tatt $n=5$ uavhengige prøver av ulik størrelse.\\
					Sett opp en likelihoodfunksjon (sannsynlighetsmaksimeringsfunksjon) for de ukjente parameterne $\mu$ og $\sigma^2$.\\
					Vis as sannsynlighetsmaksimeringsfunksjonen for $\mu$ og $\sigma^2$ blir
					\maths{&\hat\mu=\frac{\sum_{i=1}^nl_iY_i}{\sum_{i=1}^nl_i},
						&\text{og}&
						&\hat\sigma^2=\frac{\sum_{i=1}^nl_i(Y_i-\hat\mu)^2}{n}}.\vspace{-3mm}}
				{\maths{L(\mu, \sigma^2)
						=\prod_{i=1}^nf(x_i;\mu,\sigma_i)
						=\prod_{i=1}^n\frac{1}{\sqrt{2\pi}\frac{\sigma}{\sqrt l_i}}\exp\left(-\frac{(y_i-\mu)^2}{2\frac{\sigma^2}{l_i}}\right)\\
						=\prod_{i=1}^n\frac{\sqrt l_i}{\sqrt{2\pi}\sigma}\exp\left(-\frac{l_i(y_i-\mu)^2}{2\sigma^2}\right)\\
						=\frac{1}{(\sqrt{2\pi})^n}\frac{1}{(\sqrt{\sigma^2})^n}\exp\left(-\frac{1}{2\sigma^2}\sum_{i=1}^nl_i(y_i-\mu)^2\right)\prod_{i=1}^n\sqrt{l_i}}
					Finner sannsynlighetsmaksimeringsfunksjonen på vanlig måte:
					\maths{\ln L&= -\frac{n}{2}\ln(2\pi)-\frac{n}{2}\ln(\sigma^2)-\frac{1}{2\sigma^2}\sum_{i=1}^nl_i(y_i-\mu)^2+\frac{1}{2}\ln\left(\sum_{i=1}^nl_i\right)&\\
						\frac{\partial\ln L}{\partial\mu}&=-\frac{1}{2\sigma^2}\sum_{i=1}^n 2l_i(y_i-\mu)(-1)=0 \qquad\qquad\text{(kjerneregelen)}\\
						&\Rightarrow \mu\sum_{i=1}^nl_i=\sum_{i=1}^nl_iy_i
						\Rightarrow\hat\mu=\frac{\sum_{i=1}^nl_iY_i}{\sum_{i=1}^nl_i}\\
						\frac{\partial \ln L}{\partial \sigma^2}&=-\frac{n}{2\sigma^2}+\frac{1}{2\sigma^4}\sum_{i=1}^nl_i(y_i-\mu)^2=0\\
						&\Rightarrow \frac{1}{\sigma^2}\sum_{i=1}^nl_i(y_i-\mu)^2=n\\
						&\Rightarrow \hat\sigma^2=\frac{\sum_{i=1}^nl_i(Y_i-\hat\mu)^2}{n}
					}.}\vspace{-4mm}
				\deloppgave{Vis at $\hat\mu$ er forventningsrett. Vis at varinsen til $\hat\mu=\frac{\sigma^2}{\sum_{i=1}^nl_i}$\\
					Vis at $\hat\sigma^2$ ikke er forventningsrett. Du kan bruke at $\frac{n\cdot \hat\sigma^2}{\sigma^2}\sim\chi_{n-1}^2$.\\
					Hva blir en forventningsrett estimator for $\sigma^2$?}
				{\maths{E(\hat\mu)&=E\left(\frac{\sum_{i=1}^nl_iY_i}{\sum_{i=1}^nl_i}\right)=\frac{\sum_{i=1}^nl_iE(Y_i)}{\sum_{i=1}^nl_i}=\frac{\sum_{i=1}^nl_i\mu}{\sum_{i=1}^nl_i}=\mu\frac{\sum_{i=1}^nl_i}{\sum_{i=1}^nl_i}=\mu\\
						Var(\hat\mu)&=Var\left(\frac{\sum_{i=1}^nl_iY_i}{\sum_{i=1}^nl_i}\right)=\frac{\sum_{i=1}^nl_i^2Var(Y_i)}{\left(\sum_{i=1}^nl_i\right)^2}=\frac{\sum_{i=1}^nl_i^2\frac{\sigma^2}{l_i}}{\left(\sum_{i=1}^nl_i\right)^2}\\
						&=\frac{\sigma^2\sum_{i=1}^nl_i}{\left(\sum_{i=1}^nl_i\right)^2}=\frac{\sigma^2}{\sum_{i=1}^nl_i}\\
						E(\hat\sigma^2)&=E\left(\frac{n\sigma^2\hat\sigma^2}{n\sigma^2}\right)=\frac{\sigma^2}{n}E\left(\frac{n\hat\sigma^2}{\sigma^2}\right)=\sigma^2\frac{n-1}{n}\\
						S^2&=\frac{\sum_{i=1}^nl_i(Y_i-\hat\mu)^2}{n-1}}.}
			\end{block}
		\end{column}
		\begin{column}{.33\linewidth}
			\begin{block}{}
				\oppgave{Vi skal se på rekkevidde for en elektrisk bilmodell (kjørelengde fra betteriet er fullt til tomt).
					Det blir påståt at bilprodusenter oppgir urealistisk lang rekkevidde, derfor lar vi $n=15$ tilfeldig valgte sjåfører bruke bilen til normal kjøring til batteriet er tomt, for hver sjåfør er kjørelengda (km) registrert.
					Vi går ut ifra at disse kjørelengdene er et tilfeldig utvalg fra en populasjon med forventning $\mu$ ($\hat \mu=201.0667$) og varians $\sigma^2$ ($\hat \sigma^2 = 111.0667$).
					Produsenten av bilmodellen påstår er forventa rekkevidde er under tilsvarende forhold $210$ (km).
					Vi vil undersøke om det er grunnlag for å påstå at en reell rekkevidde er lavere enn dette.\\
					Sett opp hypoteser og testobservator, og utfør en test for problemstillinga over med signifikansnivå $5\%$.\\
					Finn både forkastningsområde og (tilnærma) p-verdi.
					Hva blir konklusjonen?}
				{Vi går ut ifra at observatorene i et tilfeldig utvalg $X_1,\ldots,X_{15}$ fra en normalfordelt populasjon med forventning $\mu$ og standardavvik $\sigma$.
					Vi vil teste hypotesene:
					\maths{&H_0: \mu=210 &H_1:\mu<210}
					Ettersom $\sigma$ er ukjent og maa estimeres fra data ved $S$ blir dette en T-test, testobservator:
					\maths{T=\frac{\overline X - \mu_0}{\frac{S}{\sqrt n}}}
					som er \textit{t-fordelt} med $n-1=14$ firhetsgrader når $H_0$ er sann.
					Insett data:
					\maths{t=\frac{201.067-210}{\frac{10.539}{\sqrt n}}=-3.282}
					Forkaster $H_0$ om $t<t_{0.02,14}=-1.761$. Så $H_0$ blir forkasta, og vi kan påstå at forventnia rekkevidde $\mu$ er under 210km.\\
					Kan finne \textit{p-verdien} fra:
					\maths{\text{p-verdi}=P(T_{14}<-3.282)=\begin{cases}
							<0.005\\
							>0.001
						\end{cases}}
					Så fra tabellen kan vi se at p-verdien er mellom $0.1\%$ og $0.5\%$.
					Da dette er lavere enn signifikansnivået på $0.5\%$ blir $H_0$ forkasta.}
				\deloppgave{Vi er og interresert i hvor mye rekkevidda kan variere.\\
					Utled et $90\%$-konfidensintervall for variansen $\sigma^2$.\\
					Finn intervallestimatet fra observatorene. Hva kan du konkludere fra intervallet?}{Utleder et $90\%$-konfidensintervall fra variansen, $\sigma^2$, bruker at vi kjenner fordelinga for den standariserte observatoren
					\maths{\frac{(n-1)S^2}{\sigma^2}\sim \chi_{n-1}^2&\\
						P\left(\chi_{0.95}^2<\frac{(n-1)S^2}{\sigma^2}<\chi_{0.05}^2\right)&=0.90\\
						P\left(\frac{\chi_{0.95}^2}{(n-1)S^2}<\frac{1}{\sigma^2}<\frac{\chi_{0.05}^2}{(n-1)S^2}\right)&=0.90\\
						P\left(\frac{(n-1)S^2}{\chi_{0.05}^2}<\sigma^2<\frac{(n-1)S^2}{\chi_{0.95}^2}\right)&=0.90}
					Intervallestimat fra observasjonene og $n-1=14$ frihetsgrader
					\maths{\left(\frac{(n-1)\cdot s^2}{\chi_{0.05}^2},\frac{(n-1)\cdot s^2}{\chi_{0.05}^2}\right) = \left(\frac{(15-1)\cdot 111.067}{23.685},\frac{(15-1)\cdot 111.067}{6.571}\right)\\=(65.65,236.65)}
					Med $90\%$ sannsynlighet for at intervallet inneholder den sanne verdien for variansen $\sigma^2$.
					Merk at fra dette kan vi og lage et $90\%$-konfidensintervall for standardavviket $\sigma$:
					\matte{(\sqrt{65.65},\sqrt{236.65})=(8.10,15.38)}
				}
				\deloppgave{Vi ønsker og å sammenligne modellem med en annen bilmodemm.
					Vi lar ei anna grupe $n_2=15$ sjåfører bruke denne andre modellen. Resultater: $\hat \mu=184.467, \hat \sigma^2=115.124$. Vi går ut ifra at variansen for rekkevidde er lik for de to ulike populasjonene.\\
					Gør de nødvendige forutsetningene for de to utvalga, og regn ut et estimert $95\%$-konfidensintervall for forskjell i forventa rekkevidde mellom de to modellene.
					Hva kan du konkludere fra dette intervallet?}
				{Vi går ut ifra at vi har to tilfeldige utvalg $X_{1,1},\ldots,X_{1,15}$ fra en normalfordelt populasjon med forventning $\mu_1$, og $X_{2,1},\ldots,X_{2,15}$ fra en normalfordelt populasjon med forventning $\mu_2$.
					Vi går og utifra at de to utvalga er uavhengige.
					I teksten er det gjort en anntakelse om at variansen $\sigma^2$ er den samme for de to populasjonene.\\
					Et $95\%$-konfidensintervall for $\mu_1-\mu_2$ er da gitt ved
					\maths{\left(\ol X_1 -\ol X_2-t_{0.025}\cdot S_p\cdot\sqrt{\frac{1}{n_1}+\frac{1}{n_2}},\ol X_1 -\ol X_2+t_{0.025}\cdot S_p\cdot\sqrt{\frac{1}{n_1}+\frac{1}{n_2}} \right)}
					Estimatet for den felles variansen blir
					\maths{s_p^2=\frac{s_1^2\cdot(n_1-1)+s_2^2\cdot(n_2-1)}{n_1+n_2-2}=\frac{111.067\cdot 14 + 115.124 \cdot 14}{15+15-2}}
					Dermed blir intervallestimatet med insatte observasjoner:
					\maths{201.067-194.467\pm2.048\cdot\sqrt{113.095}\cdot\sqrt{\frac{1}{15}+\frac{1}{15}}=(-1.353,14.553)}
					Her er det brukt $n_1+n_2-2=28$ frihetsgradet i t-fordelinga.\\
					Vi kan med $95\%$ sannsynlighet konkludre at den sanne forventa rekkeviddedifferansen ligger innenfor intervallgrensene.
					Da $0$ er innafor intervallet vet vi og at en tosidig test ($5\%$ signifikansnivå) ikke ville forkasta $H_0$ om ingen forskjell, kan ikke konkludere at forventa rekkevidde er ulike.
				}
			\end{block}
		\end{column}
	\end{columns}
\end{frame}
\begin{frame}{}
	\begin{columns}[t]
		\begin{column}{.50\linewidth}
			\begin{block}{Løsningsforslag}
				\oppgave{Ola samler spillekort.
					Ønsker et spesifikt spillekort ($A$).
					$X$ er anntall kort han kjøper før han får dette.
					Det er $10$ forskjellige spillekort i serien.
					Han kjøper pakker av kort med $n=20$ kort i hver pakke.
					Hver pakke er uavhengig.\\
					Hva er fordelinga for antall kort i pakka som er med spiller $A$?\\
					Hva er sannsynligheten for at Ola får minst ett kort med spiller $A$ i pakka?\\
					Hva er forventa antall ulike spillekort i pakka?}
				{Ola gjør $n=20$ kjøp.
					Resultatet av hvert kjøp er uavhengige, og sannsynligheten for suksess (kort $A$) er lik $\frac{1}{10}$ for hvert kjøp.
					Da blir $Y=$ tall på kort $A$, binomisk fordelt med $n=20$ delforsøk og sannsynlighet $p=\frac{1}{10}$:\\
					$Y \sim \text{binom}\left(n=20,p=\frac{1}{10}\right)$\\
					$P(Y\geq 1) = 1-P(X=0)=1-\binom{20}{0}\left(\frac{1}{10}\right)^0\left(1-\frac{1}{10}\right)^{20}=1-0.122=0.878$\\
					Definerer korttypene $i\in \{A,B,\cdots,J\}$:\\
					$I_i=\begin{cases}
						1, &\text{minst ett kort av typen $i$.}\\
						0, &\text{ingen kort av typen $i$}
					\end{cases}$\\
					Da blir \\
					$E(I_i)=1\cdot P(I_i=1)+0\cdot P(I_i=0)=P(I_i=1)=P(Y\geq 1)=0.878$\\
					La tallet på ulike kort være $N=\sum_i I_i$. Da blir\\
					$E(N)=E(\sum_i I_i)=\sum_i E(I_i)=\sum_i 0.878 = 10 \cdot 0.878 = 8.78$ }
				\oppgave{Fordelinga til en kontinuerlig stokastisk variabel $X$ er gitt ved følgende sannsynlighetstetthetsfunksjon:
					\maths{f(x)=\begin{cases}
							\frac{2}{9}x,&0<x<3\\
							0,&\text{ellers}
						\end{cases}}
					Vi er interresert i følgende hendelser:
					\maths{&X=\{X>1\} &\text{og}& &B=\{X>2\}}
					er disse hendelsene disjunkte?\\
					Hva er sannsynligheten for $B$?\\
					Hva er sannsynligheten for $B$ gitt $A$?\\
					Er disse hendelsene uavhengige?}
				{Nei, $A$ og $B$ er ikke disjunkte, det er mulig at $X>1$ sammtidig som $X>2$ ($X=2.5$)
					\maths{P(B)=P(X>2)=\int_2^3f(x)\,dx=\int_2^3\frac{2}{9}x\,dx\left[\frac{1}{9}x^2\right]_2^3=1-\frac{4}{9}=\frac{5}{9}\\
						P(B|A)=\frac{P(B\cap A)}{P(A)}=\frac{P(A|B)P(B)}{P(A)}=\frac{1\cdot P(B)}{P(A)}=\frac{\frac{5}{9}}{\int_1^3\frac{1}{9}x\,dx}=\frac{5}{8}}
					Vi ser at $P(B|A)\neq P(B)$, så $A$ og $B$ er ikke uavhengige}
				\oppgave{Vi vil tilpasse en lineær regresjonsmodell for $Y=$ dager, og $x=$ år, det vil si
					\maths{Y_i=\beta_0+\beta_1x_i+\epsilon_i,& &i=1,\ldots,48}
					der vi går ut ifra ar feiledda $\epsilon_1,\ldots,\epsilon_{48}$ er uavhengige og normalfordelte med forventning $0$ og varians $\sigma^2$.\\
					Regresjonsanalysen er gjennomført i $R$:\\
					\texttt{summary(lm(y~x))-> (Intercept)=$b_0$, x=$b_1$}\\
					\texttt{mean(x)=24.5, sum((x-mean(x))\^\space2)=9212}\\
					\texttt{res=residuals(ln(y~x)), sum(res\^\space2)/(48-2)=150.1756}\\
					Skriv ned minste kvadrat-estimatoren for $\beta_0, \beta_1$. og finn estimata fra $R$ (ish) utskrifta. Gi ei presis tolkning av hva de estimerte verdiene forteller deg om sammenhengen.\\
					Bruk den estimerte regresjonslinja til å finne et predikert verdi for dager i år 49}
				{\maths{B_1=\frac{\sum_{i=1}^n(x_i-\ol x)\cdot Y_i}{\sum_{i=1}^n(x_i-\ol x)^2}, &&B_0=\ol Y-B_1\cdot \ol X}
					Fra $R$-utskrifta kan vi finne MKM-estimata for $\beta_1,\beta_0$, og den estimerte regresjonslinja:
					\maths{b_1=-0.411, &&b_0=67.175 &\Rightarrow& \hat y = b_0 + b_1 x=67.175-0.411x}
					$b_0=$ skjæring med $y$-aksen, estimert forventa dager ved x=0 år.\\
					$b_1=-0.411$ er stigningstallet, estimert endring i firventa dager ekstra for hvert år.\\
					Predikert verdi for $x_0=49$ er:
					\maths{\hat y_0 = b_0 + b_1x_0=67.175-0.411\cdot x=47.036}.}
				\deloppgave{Vi er spesielt interresert i den totale endringa i antall dager som har skjedd fra $x=1$ til $x=48$. Som estimator for endring i forventa dager i løpet av de $47$ åra skal vi nytte:
					\maths{\Delta\hat Y=\hat Y_{48} - \hat Y_1}
					(Her er $\hat Y_i = B_0 + B_1 x$, altså estimert regresjonslinje år $x_i$)\\
					Vis at $\Delta\hat Y=(48-1)\cdot B_1$, der $B_1$ er estimatoren for stigningstallet.\\
					Vis at $\Delta\hat Y$ er en forventningsrett estimator for endringa over 47 år, og finn variansen til estimatoren.\\
					Bruk dette som utgangspunkt til å utlede et $95\%$-konfidensintervall for forventa endring, og estimer intervallet ved hjelp av utskriftene over.\\
					Kan du utifra dette intervallet påstå signifikant endring i antall dager?}{Differansen i estimert regresjonslinje mellom $xz=48$ og $x=1$:
					\maths{\Delta\hat Y&= \hat Y_{48}-\hat Y_1 = B_0+B_1\cdot 48-(B_0+B_1\cdot 1)=47 B_1\\
						E(\Delta\hat Y)&=E(47\cdot B_1)=47\cdot E(B_1)=47\cdot \beta_1\\
						Vet(\Delta\hat Y)&= Var(47\cdot B_1)=47^2\cdot Var(B_1)=47^2\frac{\sigma^2}{\sum_{i=1}^n(x_i-\ol x)^2}}
					Her er det vrukt at vi vet at for stigningstallestimatoren er
					\maths{E(B_1)=\beta_1&&\text{og}&&Var(B_1)=\frac{\sigma^2}{\sum_{i=1}^n(x_i-\ol x)^2}}
					Merk at forventningsverdien til estimatoren er lik forventa endring over $47$ år da
					\maths{\Delta\mu_Y=\mu_{Y|48}-\mu_{Y|1}=\beta_0+\beta_1\cdot 48-(\beta_0 + \beta_1 \cdot 1)=47\cdot\beta_1}
					Da vi vet at estimatoren $B_1$ er normalfordelt vil og $\Delta\hat Y$ bli det, og vi har at
					\maths{\Delta\hat Y\sim N\left(\Delta\mu_Y,\frac{47\sigma}{\sqrt{\sum_{i=1}{48}(x_i-\ol x)^2}}\right)}
					Dermed kan vi utlede et $95\%$-KI ved å ta utgangspunkt i en t-fordelt observator
					\maths{P\left(-t_{0.025}<\frac{\Delta\hat Y-\Delta\mu_Y}{\frac{47\cdot S}{\sqrt{\sum_{i=1}^n(x_i-\ol x)^2}}}<t_{0.025}\right)=0.95\\
						P\left(\Delta\hat Y-t_{0.025}\frac{47\cdot S}{\sqrt{\sum_{i=1}^n(x_i-\ol x)^2}}<\Delta\mu_Y<\Delta\hat Y+t_{0.025}\frac{47\cdot S}{\sqrt{\sum_{i=1}^n(x_i-\ol x)^2}}\right)=0.95\\}.
			Med $n-2=46$ frihetsgrader (bruker $t_{0.025,40}=2.021$) blir et $95\%$-intervallestimat for $\Delta\mu_Y$:
				\maths{47b_1\pm t_{0.025,46}\frac{47s}{\sqrt{\sum_{i=1}^48(x_i-\ol x)^2}}\approx 47 \cdot (-0.411)\pm 2.021 \cdot \frac{47\cdot 12.25}{\sqrt{9212}}=(-31.440,-7.193)}
				Fra $R$: $x=12.25$ og $\sum_{i=1}^{48}(x_i-\ol x)^2=9212$. Alternativt: $\hat{SE}(B_1)=\frac{s}{\sqrt{\sum_{i=1}^n(x_i-\ol x)^2}}$. Det er $95\%$ sannsynlighet for at intervallet skal inneholde forventa endring over $47$ år. Og da intervallet ikke dekker $0$ kan vi påstå med $5\%$ signifikansnivå at det har skjedd ei endring.
			}
			\end{block}
		\end{column}
		\begin{column}{.50\linewidth}
			\begin{block}{}
				\vspace{150mm}
			\end{block}
			\begin{block}{\center You can do it!}
				\center
				I belive in you!\\
				JUST DO IT!\\
				(;
			\end{block}
		\end{column}
		%\begin{column}{.33\linewidth}
		%\end{column}
	\end{columns}
\end{frame}{}
\end{document}
